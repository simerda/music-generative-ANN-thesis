\addcontentsline{toc}{chapter}{Conclusion}
\markboth{Conclusion}{Conclusion}


This thesis aimed to research \textit{music theory} basics and current approaches to music generation and state-of-the-art \textit{NLP} techniques.
We then used this theoretical knowledge to design and implement a working \textit{machine learning} model that would be able to generate music from scratch or create a continuation for an existing part of a musical piece.

Firstly we tried to use the \textit{Transformer} model for this task.
However, we soon discovered that the vanilla Transformer model is not well suited for music generation because we found out that the model only generates blank compositions.
We tried to experiment with multiple tokenization methods to whether that would solve the problem.
However, the problem prevailed.
We then tried to employ the \textit{Music Transformer} for this task, and even though it had worse accuracy (25.13 \%) than the vanilla Transformer (30.34 \%), it performed subjectively better and did not exploit blank composition generation.

We attach trained models, Python notebooks used for model training, and a usage example along with generated pieces in the attachment of this work.
