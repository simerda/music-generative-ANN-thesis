% Do not forget to include Introduction
%---------------------------------------------------------------
% \chapter{Introduction}
% uncomment the following line to create an unnumbered chapter
\chapter*{Introduction}\label{ch:introduction}
\addcontentsline{toc}{chapter}{Acknowledgments}
\markboth{Introduction}{Introduction}
%---------------------------------------------------------------
\setcounter{page}{1}

From the beginnings of modern humanity to this day and age, music has played an ever-present part in our lives.
Up until the upswing of the computer era, music composing was carried out exclusively by humans.
However, after scientists developed the first computers, people became interested in whether computers could also perform complex and creative tasks like self-driving or music generation.

\textit{Artificial intelligence} has crept into our lives more than ever before in recent years.
From fraud detection, personalization, and content recommendation to image enhancement and facial recognition, AI helped drive all those things forward.
The inventions in artificial intelligence and machine learning techniques, along with advances in computer hardware performance, helped tremendously with the ability to perform complex tasks like those mentioned above.

Even music did not escape this evolution, and there have been attempts at using \textit{RNN}s, \textit{GRU}s, and \textit{LSTM}s for automated music generation with promising results.
With the \textit{Transformer} being one of the newer models, not many papers exist on utilizing it for music composition, in contrast to recurrent neural networks.

This thesis aims to research possible ways of generating musical compositions using \textit{artificial neural networks}.
Specifically, this work focuses on leveraging neural network architectures used in \textit{natural language processing} (GRU, LSTM, Transformer, \ldots).
The goal of the practical part of this work is to create a functioning music generation model that would be able to generate songs from scratch or generate continuation for an existing song.
This solution could help music composers with the creative part of music composition.

